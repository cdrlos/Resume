\message{ !name(cs-resume.tex)}\documentclass[11pt]{res}
% \usepackage[lighttt]{lmodern}
\usepackage{amsmath,amssymb}
\usepackage[hidelinks]{hyperref}
\usepackage{gentium}
\usepackage[varqu]{zi4}
\usepackage{fontawesome}
\newsectionwidth{0pt}  % So the text is not indented under section headings

\begin{document}

\message{ !name(cs-resume.tex) !offset(-3) }

\vspace*{-50pt}
\name{\LARGE Carlos Salinas\\[12pt]} % the \\[12pt] adds a blank line after name

\address{7507 Camelot Dr. \\ Mission, TX 78572\\
  \faPhone~\href{tel:17653379606}{765-337-9606}}%
\address{\small
  \faEnvelope~\texttt{\href{mailto:cemiliosal@gmail.com}{cemiliosal@gmail.com}, \href{mailto:salinac@purdue.edu}{salinac@purdue.edu}}\\
  \small\faLinkedin~\url{https://www.linkedin.com/in/carlos-salinas-64588b160}\\
  \small\faGithub~\url{https://github.com/cdrlos}}%
\begin{resume}%

  %%%% NEW SECTION
  \section{EDUCATION}
  \emph{Master of Science,} Mathematics \hfill August 2020\\
  Purdue University, West Lafayette, IN\\ GPA 3.73\\[2pt]
  \emph{Bachelor of Science,} Mathematics\hfill May 2014\\
  University of Texas--Pan American, Edinburg, TX

  %%%% NEW SECTION
  \vspace{-10pt}
  \section{SKILLS}
  \emph{Programming languages}: Scheme, Common Lisp, Python, R, C, Java, Octave/MATLAB \\
  \emph{Python packages}: numpy, scikit-learn, matplotlib, pandas, keras, spaCy\\
  \emph{R packages}: ggplot, shiny, tidyverse, dplyr, forcats, modelr, broom\\
  \emph{Misc.\ software}: Emacs, Vim, SQL, Git, Latex, LibreOffice, Google Docs\\
  \emph{Operating systems}: Linux (Debian, Fedora), FreeBSD, Windows\\
  \emph{Natural languages}: English, Spanish, Russian, French, Persian/Farsi

  %%%% NEW SECTION
  \vspace{-10pt}
  \section{EXPERIENCE}
  \emph{Teaching Assistant}\hfill August 2014-May 2020\\
  Department of Mathematics, Purdue University, West Lafayette, IN
  \begin{itemize} \itemsep -3pt
    \item Held recitation for Purdue's various calculus 1, 2, and 3 courses.
    \item Graded differential equations and linear algebra courses at the
      undergraduate and graduate level.
    \item Instructor for undergraduate calculus 2.
  \end{itemize}
   \vspace{-10pt}
  \emph{Research Assistant} \hfill January-May 2014 \\
  University of Texas--Pan American, Department of Mathematics,
  Edinburg, TX
  \begin{itemize} \itemsep -3pt %reduce space between items
    \item Studied $2$-special pairs in the free group on two generators under the supervision of Dr.\ S.D.\ Lawton.
    \item Discovered a correspondence between the number of $2$-special pairs and pairs of orientable necklaces.
    \item Published the associated sequence in the On-line Encyclopedia of Integer Sequences under \href{https://oeis.org/A237623}{A237623}.
  \end{itemize}
  \vspace{-10pt}
  \emph{EAGL System Admin} \hfill  September 2013-May 2014\\
  University of Texas--Pan American, Department of Mathematics,
  Edinburg, TX
  \begin{itemize} \itemsep -3pt %reduce space between items
    \item Admin for the department's Experimental Algebra \& Geometry Lab's Fedora cluster.
    \item Operated and updated EAGL's
    \item Assisted Dr.\ S.D.\ Lawton with various mathematics outreach activities in the community.
  \end{itemize}
  % \emph{Summer Research Opportunity Program} \hfill June-August 2010 \\
  % Massachusetts Institute of Technology,
  % DMSE,
  % Cambridge, MA
  % \begin{itemize} \itemsep -3pt %reduce space between items
  %   \item Studied elasticity modulus of Ti-Ta shape-memory alloy.
  %   \item Experimented with several annealing and alloy processing
  %     methods to obtain better shape-memory properties.
  % \end{itemize}
  %%%% NEW SECTION
  % \vspace{-10pt}
  % \section{RELEVANT COURSEWORK}
  % \emph{Wavelet image compression} \hfill
  % September-December 2015\\
  % Purdue University, Department of Mathematics, West Lafayette, IN
  % \begin{itemize} \itemsep -3pt
  %   \item Debugged and documented C code for wavelet image compression written by Dr.\ B.J.\ Lucier.
  %   \item Reworked the Q-coder which is an adaptive algorithm that compresses
  %   binary sequences.
  % \end{itemize}

  %%%% NEW SECTION
  \vspace{-10pt}
  \section{TALKS}
  \emph{The Black--Scholes model as an application of Itô calculus}\hfill
  Summer 2019\\
  Student Analysis Seminar, Department of Mathematics, Purdue University
  \begin{itemize} \itemsep -3pt
    \item Introduced the audience to the Black--Scholes model for option pricing and solved it using the Feymann--Kac formula.
  \end{itemize}
  \vspace{-10pt}
  \emph{Cybenko's universal approximation theorem}\hfill Spring 2019\\
  Machine Learning
\message{ !name(cs-resume.tex) !offset(31) }

\end{document}

%%% Local Variables:
%%% mode: latex
%%% TeX-master: t
%%% End:
