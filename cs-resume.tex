\documentclass{res}
\usepackage[hidelinks]{hyperref}
\usepackage{stix2}
\usepackage{inconsolata}
\usepackage{fontawesome}

\addtolength{\topmargin}{-.75in}
\addtolength{\textheight}{1.75in}
\renewcommand{\baselinestretch}{0.95}
\newsectionwidth{0pt}  % So the text is not indented under section headings

\begin{document}
\name{\LARGE Carlos Salinas\\[5pt]} % the \\[12pt] adds a blank line after name
\address{\small
  \setlength{\tabcolsep}{2pt}
  \begin{tabular}{ll}
    Address:&7507 Camelot Dr, Mission, TX 78572\\
    Mobile:&765-337-9606\\
    E-Mail:&\texttt{cemiliosal@gmail.com}
    \end{tabular}
}
\begin{resume}%
  \vspace{-5pt}
  \section{EDUCATION}
  \emph{Master of Science,} Mathematics \hfill August 2020\\
  Purdue University, West Lafayette, IN\\ GPA 3.73\\[3pt]
  \emph{Bachelor of Science,} Mathematics\hfill May 2014\\
  University of Texas--Pan American, Edinburg, TX\\ GPA 3.86
  %%%% NEW SECTION
  \vspace{-5pt}
  \section{SKILLS}
  \emph{Programming languages}: Lisp, Python, R, C, Octave/MATLAB, Java \\
  \emph{Python packages}: numpy, scikit-learn, matplotlib, pandas, keras, spaCy\\
  \emph{R packages}: ggplot, tidyverse, tidymodels, dplyr, forcats, modelr, shiny\\
  \emph{Misc.\ software}: Emacs, Vim, SQL, Git, Latex, Mathematica, LibreOffice, Google Docs\\
  \emph{Operating systems}: Linux, FreeBSD, Windows\\
  \emph{Languages}: English (Native), Spanish (Native), Russian (Fluent), Persian (Conversant)
  %%%% NEW SECTION
  \vspace{-5pt}
  \section{RELEVANT EXPERIENCE}
  \emph{Teaching Assistant}\hfill August 2014-May 2020\\
  Department of Mathematics, Purdue University, West Lafayette, IN
  \begin{itemize} \itemsep -3pt
    \item Lead several recitation classes for undergraduate level calculus 1, 2, and 3.
    \item Lectured two calculus 2 classes.
    \item Recorded and analyzed student's scores using tools such as MS Excel.
    \item Graded differential equations and linear algebra at the undergraduate and graduate level.
    \item Maintained a university-associated
      % \href{https://www.math.purdue.edu/~salinac}
      {website} where course
      material such as notes and solutions were uploaded.
    % \item Instructed undergraduate calculus 1, 2, and 3.
    % \item Analyzed and assigned student's cummulative work.
    % \item Graded differential equations and linear algebra courses at the
    %   undergraduate and graduate level.
    % \item Uploaded notes and solutions to course related material to university associated
    %   \href{https://www.math.purdue.edu/~salinac}{website}.
  \end{itemize}
  \vspace{-7pt}
  \emph{Research Assistant} \hfill September 2013-May 2014 \\
  University of Texas--Pan American, Department of Mathematics,
  Edinburg, TX
  \begin{itemize} \itemsep -3pt %reduce space between items
    % \item Studied $2$-special pairs in the free group on two generators under the supervision of Dr.\ S.D.\ Lawton.
    \item Wrote algorithms in Mathematica to study the trace of representations in character varieties.
    \item Discovered a correspondence between so-called $2$-special pairs and pairs of
      orientable necklaces.
    \item Published the associated sequence in the On-line Encyclopedia of
      Integer Sequences under % \href{https://oeis.org/A237623}
      {A237623}.
      \item Presented results at Howard University's Workshop on Character Varieties and Geometric Structures.
  \end{itemize}
  \vspace{-7pt}
  \emph{Experimental Algebra and Geometry Lab System Admin} \hfill  September 2013-May 2014\\
  University of Texas--Pan American, Department of Mathematics,
  Edinburg, TX
  \begin{itemize} \itemsep -3pt %reduce space between items
    \item Operated the department's Experimental Algebra and Geometry Lab.
    \item Maintained an operational CUDALink computing workstation for doing Mathematica simulations.
    \item Managed the lab's 3D printer.
    \item Engaged with local schools in math and geometry related outreach activities.
  \end{itemize}
  % \emph{Summer Research Opportunity Program} \hfill June-August 2010 \\
  % Massachusetts Institute of Technology,
  % DMSE,
  % Cambridge, MA
  % \begin{itemize} \itemsep -3pt %reduce space between items
  %   \item Studied elasticity modulus of Ti-Ta shape-memory alloy.
  %   \item Experimented with several annealing and alloy processing
  %     methods to obtain better shape-memory properties.
  % \end{itemize}
  %%%% NEW SECTION
  % \vspace{-7pt}
  % \section{RELEVANT COURSEWORK}
  % \emph{Wavelet image compression} \hfill
  % September-December 2015\\
  % Purdue University, Department of Mathematics, West Lafayette, IN
  % \begin{itemize} \itemsep -3pt
  %   \item Debugged and documented C code for wavelet image compression written by Dr.\ B.J.\ Lucier.
  %   \item Reworked the Q-coder which is an adaptive algorithm that compresses
  %   binary sequences.
  % \end{itemize}

  %%%% NEW SECTION
  \vspace{-5pt}
  \section{TALKS}
  \emph{Trends in book-reading over the years}\hfill Summer 2020\\
  PI4 Computational Boot Camp, University of Illinois Urbana-Champaign
  \begin{itemize}\itemsep -3pt
    \item Wrote R code together with a team to analyze and predict trends in book-reading.
    \item Modeled the \emph{popularity} of well-known authors using R's modelr
      package.
  \end{itemize}
  \vspace{-7pt}
  \emph{The Black--Scholes model as an application of Itô calculus}\hfill
  Summer 2019\\
  Student Analysis Seminar, Department of Mathematics, Purdue University
  \begin{itemize} \itemsep -3pt
    \item Introduced the audience to the Black--Scholes model and solved it using the Feymann--Kac formula.
  \end{itemize}
  \vspace{-7pt}
  \emph{Cybenko's Approximations by superpositions of sigmoidal functions}\hfill Spring 2019\\
  Machine Learning and Information Processing Reading Group, Purdue University,  West Lafayette, IN
  \begin{itemize} \itemsep -3pt
    \item Introduced the audience to a foundational result in the study of artificial neural networks.
    \item Proved the necessary lemmas to show that sigmoidal functions can approximate any continuous function.
  \end{itemize}
  \vspace{-7pt}
  \emph{The Bott periodicity theorem}\hfill Fall 2017\\
  Student Colloquium Department of Mathematics, Purdue University, West Lafayette, IN
  \begin{itemize} \itemsep -3pt
    \item Proved the Bott periodicity theorem from the perspective of classifying space theory.
  \end{itemize}
  \vspace{-7pt}
  \emph{Wavelet image compression} \hfill
  Fall 2015\\
  Department of Mathematics, Purdue University, West Lafayette, IN
  \begin{itemize} \itemsep -3pt
    \item Debugged and documented C code for wavelet image compression.
    \item Gave a talk on the algorithm behind the compression code together with a team.
  \end{itemize}
  \vspace{-7pt}
  % \emph{Introduction to geometric group theory}\\
  % Department of Mathematics, Purdue University, West Lafayette, IN
  % \begin{itemize} \itemsep -3pt
  %   \item Gave a short introduction to geometric group theory with examples primarily in the modular group.
  % \end{itemize}

  % \vspace{-7pt}
  % \emph{Making graphics for your students in TikZ and Asymptote}\\
  % Department of Mathematics, Purdue University, West Lafayette, IN
  % \begin{itemize} \itemsep -3pt
  %   \item Introduced participants to TikZ/Asymptote for creating simple graphics which can be easily included in a Latex project.
  % \end{itemize}
  %%%% NEW SECTION
  % \vspace{-7pt}
  % \section{CURRENT PROJECTS}
\end{resume}
\end{document}

%%% Local Variables:
%%% mode: latex
%%% TeX-master: t
%%% End:
