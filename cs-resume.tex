\documentclass{res}
\usepackage[hidelinks]{hyperref}
\usepackage{stix2}
\usepackage{inconsolata}
% \usepackage[lighttt]{lmodern}
\usepackage{fontawesome}
\usepackage[activate={true,nocompatibility},final,tracking=true,kerning=true,spacing=true,factor=1100,stretch=10,shrink=10]{microtype}

\addtolength{\topmargin}{-.75in}
\addtolength{\textheight}{1.75in}
\renewcommand{\baselinestretch}{0.95}
\newsectionwidth{0pt}  % So the text is not indented under section headings

\begin{document}
\name{\LARGE Carlos Salinas\\[5pt]} % the \\[12pt] adds a blank line after name
\address{\small
  \setlength{\tabcolsep}{2pt}
  \begin{tabular}{ll}
    Address:&7507 Camelot Dr, Mission, TX 78572\\
    Mobile:&(765) 337-9606\\
    E-mail:&\texttt{cemiliosal@gmail.com}, \texttt{salinac@purdue.edu}
  \end{tabular}
}
\begin{resume}%
  \vspace{-5pt}
  \section{EDUCATION}
  \emph{Master of Science/ABD PhD,} Mathematics \hfill August 2020\\
  Purdue University, West Lafayette, IN\\ GPA 3.73\\[3pt]
  \emph{Bachelor of Science,} Mathematics\hfill May 2014\\
  University of Texas--Pan American, Edinburg, TX\\ GPA 3.86
  %%%% NEW SECTION
  \vspace{-5pt}
  \section{SKILLS}
  \emph{Programming languages}: Lisp, Python, C, Java, R, Matlab\\
  \emph{Python packages}: numpy, scikit-learn, matplotlib, pandas, keras, spaCy\\
  \emph{R packages}: ggplot, tidyverse, tidymodels, dplyr, forcats, modelr, shiny\\
  \emph{Misc.\ software}: Emacs, Vim, SQL, Git, Latex, Mathematica, LibreOffice, Google Docs\\
  \emph{Operating systems}: Linux (Debian, Fedora), FreeBSD, Windows\\
  \emph{Languages}: English (Native), Spanish (Native), Russian (Fluent), Persian (Conversant)
  %%%% NEW SECTION
  \vspace{-5pt}
  \section{RELEVANT EXPERIENCE}
  \emph{Teaching Assistant}\hfill August 2014-May 2020\\
  Department of Mathematics, Purdue University, West Lafayette, IN
  \begin{itemize} \itemsep -3pt
    \item Led two to three recitation sessions per semester for one of Purdue's calculus 1, 2, or 3 courses.
    \item Uploaded students' scores to web-based academic software such as BlackBoard.
    \item Analyzed students' performance on homework, quizzes, and midterms using spreadsheet software.
    \item Graded differential equations and linear algebra homework at the undergraduate and graduate level.
    \item Maintained a university associated
    % \href{https://www.math.purdue.edu/~salinac}
      {website} where to which I uploaded relevant course material such as quiz and homework solutions, recitation notes, and quiz and midterm statistics.
  \end{itemize}
  \vspace{-7pt}
  \emph{Experimental Algebra and Geometry Lab System Admin/Research Assistant} \hfill September 2013-May 2014 \\
  University of Texas--Pan American, Department of Mathematics,
  Edinburg, TX
  \begin{itemize} \itemsep -3pt %reduce space between items
    \item Administrated the lab's Fedora cluster which included installing hardware, configuring the network, and installing and updating software.
    \item Maintained an operational CUDA computing workstation for doing parallel numerical simulations.
    \item Upkept the lab's 3D printer.
    \item Engaged with local schools in math and geometry related outreach activities.
    \item Wrote algorithms in Mathematica to study the trace of representations in character varieties.
    \item Discovered a correspondence between so-called $2$-special pairs and pairs of
      orientable necklaces.
    \item Published the associated sequence in the On-line Encyclopedia of
      Integer Sequences under % \href{https://oeis.org/A237623}
      {A237623}.
    \item Presented results at Howard University's Workshop on Character Varieties and Geometric Structures.
  \end{itemize}
  % \emph{Summer Research Opportunity Program} \hfill June-August 2010 \\
  % Massachusetts Institute of Technology,
  % DMSE,
  % Cambridge, MA
  % \begin{itemize} \itemsep -3pt %reduce space between items
        %       \item Studied elasticity modulus of Ti-Ta shape-memory alloy.
        %       \item Experimented with several annealing and alloy processing
        %         methods to obtain better shape-memory properties.
        %     \end{itemize}
        %%%%     NEW SECTION
        %         \vspace{-7pt}
        %         \section{RELEVANT COURSEWORK}
        %         \emph{Wavelet image compression} \hfill
        %         September-December 2015\\
        %         Purdue University, Department of Mathematics, West Lafayette, IN
        %         \begin{itemize} \itemsep -3pt
        %       \item Debugged and documented C code for wavelet image compression written by Dr.\ B.J.\ Lucier.
        %       \item Reworked the Q-coder which is an adaptive algorithm that compresses
        %         binary sequences.
        %     \end{itemize}

        %%%%     NEW SECTION
  \vspace{-5pt}
  \section{TALKS}
  \emph{Trends in book-reading over the years}\hfill Summer 2020\\
  PI4 Computational Boot Camp, University of Illinois Urbana-Champaign
  \begin{itemize}\itemsep -3pt
    \item Wrote an RMarkdown notebook to analyze book ratings on a subset of GoodReads' database.
    \item Performed linear regression to model the popularity of prolific authors.
    \item Performed ANOVA on the data to get an overview of the different features of the data and how they relate to the \emph{popularity} of a given author.
  \end{itemize}
  \vspace{-7pt}
  \emph{The Black--Scholes model as an application of Itô calculus}\hfill
  Summer 2019\\
  Student Analysis Seminar, Department of Mathematics, Purdue University
  \begin{itemize} \itemsep -3pt
    \item Gave a short introduction to financial mathematics.
    \item Introduced the Black--Scholes model and solved it using the Feymann--Kac formula.
  \end{itemize}
  \vspace{-7pt}
  \emph{Cybenko's Approximations by superpositions of sigmoidal functions}\hfill Spring 2019\\
  Machine Learning and Information Processing Reading Group, Purdue University,  West Lafayette, IN
  \begin{itemize} \itemsep -3pt
    \item Introduced the audience to a foundational result in the study of artificial neural networks.
    \item Proved the necessary lemmas to show that sigmoidal functions can approximate any continuous function.
  \end{itemize}
  \vspace{-7pt}
  \emph{The Bott periodicity theorem}\hfill Fall 2017\\
  Student Colloquium Department of Mathematics, Purdue University, West Lafayette, IN
  \begin{itemize} \itemsep -3pt
    \item Proved the Bott periodicity theorem from the perspective of classifying space theory.
  \end{itemize}
  \vspace{-7pt}
  \emph{Wavelet image compression} \hfill
  Fall 2015\\
  Department of Mathematics, Purdue University, West Lafayette, IN
  \begin{itemize} \itemsep -3pt
    \item Debugged an basic implementation of a wavelet image compression algorithm in C.
    \item Presented functional code to the audience and explained specifics behind the compression algorithm.
  \end{itemize}
  \vspace{-7pt}
        %         \emph{Introduction to geometric group theory}\\
        %         Department of Mathematics, Purdue University, West Lafayette, IN
        %         \begin{itemize} \itemsep -3pt
        %       \item Gave a short introduction to geometric group theory with examples primarily in the modular group.
        %     \end{itemize}

        %         \vspace{-7pt}
        %         \emph{Making graphics for your students in TikZ and Asymptote}\\
        %         Department of Mathematics, Purdue University, West Lafayette, IN
        %         \begin{itemize} \itemsep -3pt
        %       \item Introduced participants to TikZ/Asymptote for creating simple graphics which can be easily included in a Latex project.
        %     \end{itemize}
        %%%%     NEW SECTION
        %         \vspace{-7pt}
        %         \section{CURRENT PROJECTS}
\end{resume}
\end{document}

%%% Local Variables:
%%% mode: latex
%%% TeX-master: t
%%% End:
